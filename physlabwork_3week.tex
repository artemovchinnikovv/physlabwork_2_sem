%---------------------------------------------------
%	PACKAGES AND OTHER DOCUMENT CONFIGURATIONS
%------------------------------------------------------------------------------------


\documentclass[1 pt]{article}
\usepackage{amsmath,amsthm,amssymb}
\usepackage{mathtext}
\usepackage[T1,T2A]{fontenc}
\usepackage[utf8]{inputenc}
\usepackage[english,russian]{babel}
\usepackage{graphicx}
\usepackage{natbib}
\usepackage{pgfplots}
\usepackage[inkscapeformat=png]{svg}
\pgfplotsset{compat=1.9}

\begin {document}

\begin{titlepage}
\newcommand{\HRule}{\rule{\linewidth}{0.3 mm}} % Defines a Hnew command for the horizontal lines, change thickness here

\center % Center everything on the page
 
%----------------------------------------------------------------------------------------
%	HEADING SECTIONS
%----------------------------------------------------------------------------------------

\textsc{\Large Московский физико-технический институт }\\[1.5cm] % Name of your university/college
\textsc{\Large Факультет аэрокосмических технологий}\\[0.5cm] % Major heading such as course name
\textsc{\large Кафедра общей физики}\\[0.5cm] % Minor heading such as course title

%----------------------------------------------------------------------------------------
%	TITLE SECTION
%----------------------------------------------------------------------------------------

\HRule \\[0.4cm]
{ \huge \bfseries Экспериментальное определение теплоты испарения спирта }\\[0.4cm] % Title of your document
\HRule \\[1.5cm]
 
%----------------------------------------------------------------------------------------
%	AUTHOR SECTION
%----------------------------------------------------------------------------------------

\begin{minipage}{0.4\textwidth}
\begin{flushleft} \large
\emph{Автор:}\\ Артем \textsc{Овчинников} % Your name
\end{flushleft}
\end{minipage}
\begin{minipage}{0.4\textwidth}
\begin{flushright} \large
\emph{Преподаватель:} \\
Арина Владимировна \textsc{Радивон} % Supervisor's Name
\end{flushright}
\end{minipage}\\[4cm]
%	DATE SECTION
%----------------------------------------------------------------------------------------

{\large \today}\\[2cm] % Date, change the \today to a set date if you want to be precise

%----------------------------------------------------------------------------------------
%	LOGO SECTION
%----------------------------------------------------------------------------------------

 
%----------------------------------------------------------------------------------------

\vfill % Fill the rest of the page with whitespace

\end{titlepage}
\tableofcontents
\newpage
\section{Аннотация}
\large В данной работе представлено экспериментальное определение теплоты испарения спирта косвенным методом, основанным на формуле Клайпейрона-Клаузиуса.
\section{Теоретические сведения}
Формула Клапейрона-Клаузиуса:
\begin{equation}
    \frac{dP}{dT} = \frac{L}{T(V_2-V_1)}
\end{equation}
P - давление насыщенного пара при температуре T, 
T - абсолютная температура жидкости и пара, 
L - теплота испарения жидкости, 
V2 - объем пара, 
V1 - объем жидкости. \\
\\
Малость объема жидкости по сравнению с объемом пара:
\begin{equation*}
    V_1 \ll V_2
\end{equation*}
\\
Уравнение Ван-дер-Ваальса:
\begin{equation}
    (P+\frac{a}{V^2})(V-b) = RT
\end{equation}
\\
При давлениях ниже атмосферного константами в уравнении Ван-дер-Ваальса можно пренебречь (газ можно рассматривать как идеальный):
\begin{equation*}
    P \gg \frac{a}{V^2}
\end{equation*}
\begin{equation*}
    V \gg b
\end{equation*}
\begin{equation}
    V = \frac{RT}{P}
\end{equation}
\\
Подставляя в формулу Клапейрона-Клаузиуса, получаем рабочую формулу:
\begin{equation}
    L = -R \frac{d(ln P)}{d(1/T)}
\end{equation}
\newpage
\section{Методика измерений}
\begin{enumerate} 
\item Выставить термостат на нужную температуру.
\item Дождаться установления термодинамического равновесия (1-3 минуты при разнице в 1 градус Цельсия).
\item Определить и записать температуру по электронному термометру.
\item Определить и записать давление по ртутному манометру.
\item Повторить.
\end{enumerate}
\section{Используемое оборудование}
Рисунок используемой экспериментальной установки:
\begin{center}
    \includesvg[]{physlabwork_3week_exp_setup}
\end{center}
12 - термостат (емкость с водой),
13 - запаянная капсула,
14 - спирт,
15 - ртутный манометр,
16 - отчетный микроскоп,
17 - шкала измерений разности высот.
\newpage
\section{Результаты измерений и обработка данных}
\begin{center}
    \includesvg[]{physlabwork_3week_plot}
\end{center}
\begin{center}
    \includesvg[]{physlabwork_3week_plot_error_points}
\end{center}
\newpage
\section{Обсуждение результатов}
\begin{equation*}
    L_{exp} = 41592 \pm 323 \frac{Дж}{моль}
\end{equation*}
\begin{equation*}
    L_{known} = 41630 \frac{Дж}{моль}
\end{equation*}
\section{Заключение}
Вычисленное значение молярной теплоты парообразования совпадает в пределах погрешности с табличным значением, однако, использованный метод вычисления погрешностей не всегда дает результат в ее пределах, что говорит о недооценке погрешностей. В данном случае первоочередной неоцененной причиной погрешности является неравномерный нагрев калориметра. Также стоит учитывать тот факт, что молярная теплота парообразования не является константой, а зависит от давления и температуры.
\end{document}
