%---------------------------------------------------
%	PACKAGES AND OTHER DOCUMENT CONFIGURATIONS
%------------------------------------------------------------------------------------


\documentclass[1 pt]{article}
\usepackage{amsmath,amsthm,amssymb}
\usepackage{mathtext}
\usepackage[T1,T2A]{fontenc}
\usepackage[utf8]{inputenc}
\usepackage[english,russian]{babel}
\usepackage{graphicx}
\usepackage{natbib}
\usepackage{pgfplots}
\usepackage[inkscapeformat=png]{svg}
\pgfplotsset{compat=1.9}

\begin {document}

\begin{titlepage}
\newcommand{\HRule}{\rule{\linewidth}{0.3 mm}} % Defines a Hnew command for the horizontal lines, change thickness here

\center % Center everything on the page
 
%----------------------------------------------------------------------------------------
%	HEADING SECTIONS
%----------------------------------------------------------------------------------------

\textsc{\Large Московский физико-технический институт }\\[1.5cm] % Name of your university/college
\textsc{\Large Факультет аэрокосмических технологий}\\[0.5cm] % Major heading such as course name
\textsc{\large Кафедра общей физики}\\[0.5cm] % Minor heading such as course title

%----------------------------------------------------------------------------------------
%	TITLE SECTION
%----------------------------------------------------------------------------------------

\HRule \\[0.4cm]
{ \huge \bfseries Экспериментальное вычисление коэффициентов a и b модели Ван-дер-Ваальса для углекислого газа}\\[0.4cm] % Title of your document
\HRule \\[1.5cm]
 
%----------------------------------------------------------------------------------------
%	AUTHOR SECTION
%----------------------------------------------------------------------------------------

\begin{minipage}{0.4\textwidth}
\begin{flushleft} \large
\emph{Автор:}\\ Артем \textsc{Овчинников} % Your name
\end{flushleft}
\end{minipage}
\begin{minipage}{0.4\textwidth}
\begin{flushright} \large
\emph{Преподаватель:} \\
Арина Владимировна \textsc{Радивон} % Supervisor's Name
\end{flushright}
\end{minipage}\\[4cm]
%	DATE SECTION
%----------------------------------------------------------------------------------------

{\large \today}\\[2cm] % Date, change the \today to a set date if you want to be precise

%----------------------------------------------------------------------------------------
%	LOGO SECTION
%----------------------------------------------------------------------------------------

 
%----------------------------------------------------------------------------------------

\vfill % Fill the rest of the page with whitespace

\end{titlepage}
\section{Аннотация}
\large В данной работе представлено экспериментальное получение коэффициентов a и b углекислого газа модели реального газа Ван-дер-Ваальса, коэффициента Джоуля-Томсона для температур: 18, 30, 50 градусов Цельсия.
\section{Теоретические сведения}
Эффектом Джоуля–Томсона называется изменение температуры газа, медленно
просачивающегося из области высокого в область низкого давления в условиях тепловой изоляции.
Разность работ при проходе через перегородку(процесс адиабатический):
\begin{equation}
    A_1-A_2 = P_{1}V_{1} - P_{2}V_{2} = (U_2+\mu v_{2}/2) - (U_1+\mu v_{1}/2)
\end{equation}
Определение энтальпии:
\begin{equation}
    H = U + PV
\end{equation}
\begin{equation}
    H_1 - H_2 = \mu (v_{2}/2 - v_{1}/2)
\end{equation}
Исходя из малости кинетической энергии относительно тепловой:
\begin{equation}
    H_1 \approx H_2
\end{equation}
Энтальпия зависит от температуры и давления:
\begin{equation}
    \mu_{JT} = \frac{\Delta T}{\Delta P}
\end{equation}
Модель реального газа Ван-дер-Ваальса:
\begin{equation}
    (P+\frac{a}{V^2})(V-b) = RT
\end{equation}
Калорическое уравнение:
\begin{equation}
    U = C_V-\frac{a}{V}
\end{equation}
Энтальпия через модель реального газа Ван-дер-Ваальса
\begin{equation}
    H = U + PV = C_{V}T + RT \frac{V}{V-b} - \frac{2a}{V}
\end{equation}
Модель идеального газа для объёма:
\begin{equation}
    V \approx \frac{RT}{P}
\end{equation}
Малость b:
\begin{equation}
    \frac{V}{V-b} \approx 1 + \frac{b}{V}
\end{equation}
\begin{equation}
    H \approx C_{P}T+P(b-\frac{2a}{RT})
\end{equation}
Малость изменения температуры:
\begin{equation}
    dT/T \ll 1
\end{equation}
\begin{equation}
    dH \approx 0
\end{equation}
Финальная формула:
\begin{equation}
    \mu_{JT} = - \frac{dT}{|dP|} \approx - \frac{b-\frac{2a}{RT}}{C_P}
\end{equation}
\section{Методика измерений}
\begin{enumerate} 
  \item Откалибровать напряжение и разность давлений (при нулевой разности давлений - нулевой разности температур может быть ненулевое напряжение).
  \item Выставить термостат на нужную температуру, дождаться достижения нужной температуры.
  \item Открыть вентиль, выставить нужное давление.
  \item Дождаться установления равновесия (разность температур - напряжение не должно меняться).
  \item Зафиксировать напряжение от разности давления при данной температуре, повторить с другими значениями.
\end{enumerate}
\newpage
\section{Используемое оборудование}
\begin{center}
    \includegraphics[scale=0.75]{2_1_6.png}
\end{center}
1 - трубка
2 - пористая перегородка
3 - труба Дьюара
4 - уплотнительное кольцо
5 - змеевик
6 - балластный баллон
7 - цифровой вольтметр
8, 9 - спаи
10 - пробка из пенопласта
\newpage
\section{Результаты измерений и обработка данных}
\begin{center}
    \includesvg[]{physlabwork_1week_plot_dt_dp_2_1.svg}
\end{center}
\begin{center}
    \includesvg[]{physlabwork_1week_plot_dt_dp_1_1.svg}
\end{center}
\begin{center}
    \includesvg[]{physlabwork_1week_plot_dt_dp_3.svg}
\end{center}
\newpage
\section{Обсуждение результатов}
\begin{equation*}
    a \approx 1 \frac{м^{6}Па}{моль^2}
\end{equation*}
\begin{equation*}
    b \approx 4*10^{-4} \frac{м^3}{моль}
\end{equation*}
\begin{equation*}
    T_{inv}  \approx 500 K
\end{equation*}
Для температуры 18 градусов Цельсия:
\begin{equation*}
    \mu_{JT} = 0.97 \pm 0.08 \frac{K}{bar}
\end{equation*}
Для температуры 30 градусов Цельсия:
\begin{equation*}
    \mu_{JT} = 0.86 \pm 0.08 \frac{K}{bar}
\end{equation*}
Для температуры 50 градусов Цельсия:
\begin{equation*}
    \mu_{JT} = 0.75 \pm 0.06 \frac{K}{bar}
\end{equation*}
\begin{equation*}
    a = 0.3658 \frac{м^{6}Па}{моль^2}
\end{equation*}
\begin{equation*}
    b = 42.9*10^{-6} \frac{м^3}{моль}
\end{equation*}
\begin{equation*}
    T_{inv} \approx 1500 K
\end{equation*}
\section{Заключение}
Данный эксперимент и данная теория недостаточно точные для определения коэффициентов газа Ван-дер-Ваальса. Потеря точности может происходить из-за многих факторов, вот одни из них:
\begin{enumerate} 
  \item Газ обмениваеться теплом с трубкой.
  \item Кинетическая энергия сравнима с тепловой.
  \item Пренебрежения при выводе формулы энтальпии.
\end{enumerate}
\newpage
\tableofcontents
\end{document}
